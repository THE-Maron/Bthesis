・プレゼンター1人,傍聴者3人によるプレゼンテーションを行った

・プレゼンと質問時間合わせて10分程度

・提案アプリケーション+OEB使用時と,Zoomのみを使用した場合の2度行った

・プレゼンター負担削減のため,プレゼン資料は事前に作成した(2種類)

・また,各傍聴者に,最低一回の質問機会を設けるため,各プレゼンに対し3つの質問項目を事前に用意した

・実験環境は以下の通り

・共通:THETA VとTHETA Sを使用,プレゼンターは全天球カメラに対し30cmほど離れていた,プレゼンターのPC詳細(後で調べる)

・Zoom使用時:質問者側はMac book pro(栗岡机上:後で詳しく)を使用(横に3人並んで一つの画面を見た)

・OEB使用時:質問者側はOmniEyeBallを使用