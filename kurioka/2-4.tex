2-1節では,全天球カメラの概要と,全天球カメラ自体の問題点について指摘した.
全天球カメラは魚眼レンズを主に使用しており,パノラマ
画像の生成時には歪みがどうしても発生してしまう.
また,全天球カメラの視野の広さは,メリットが大きいものの,従来のカメラ映像には
存在しなかった新たな問題を引き起こした.

2-2節では,全天球ディスプレイと,全天球ディスプレイを使用した研究について
紹介した.全天球ディスプレイを用いたビデオ会議システムがいくつか提案されており,
立体感という観点から,ユーザーから高い評価を受けており,これが本研究でも
全天球ディスプレイを使用する動機になっている.

2-3節では,ビデオ会議に関する研究をいくつか紹介した.
ビデオ会議では対面でのコミュニケーションで得られるはずの
情報が失われ,その情報を再現するための研究が行われている.
そのような情報は,再現した結果有効に働いており,ビデオ会議における
臨場感の向上には必要であるといえる.

以上の流れをふまえ,3章で本研究における1対多人数の
ビデオ会議システムの提案を行う.