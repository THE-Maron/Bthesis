・表5.1,5.2を見るとOEB使用時,平面ディスプレイ使用時共に真陽性率(プレゼンターが見ていた時間の中で,目が合ったと感じた時間の割合)
の最大値,最小値の差が非常に大きい(極端な例は平面ディスプレイ使用時の0.00(一度もキーを押さなかった))

・自由記述意見の1つ「発表者の顔の画像が暗くて小さいため、何となく正面を向いている時に目があっていそうだと感じた。曖昧であり、判断に困った」(平面ディスプレイ使用時)
に見られるように,そもそも「目が合っている」という感覚自体があいまいで,人によって見られていると感じた機会が不ぞろいである原因であると考えられる.

・真陽性率は3人とも平面ディスプレイ使用時に対し,OEB使用後の方が増加している

・サンプル数の少なさ,及び順序効果により,比較はできないが,本実験で使用したアプリケーションによって
視線情報が共有されやすくなったという仮定に対し,一考の余地がある.

・一方で,視線情報を小窓の表示状態などからも判断できるOEB使用時の方が,偽陽性率が下がると予想していたが,
2者はOEB使用時の方が偽陽性率が増加した.特に,被検者2の偽陽性率は0.16上昇し0.28となっている.

・「正面に発表者がいて、さらに窓に発表者がいることに違和感を感じました。」(OEB使用時)という意見があった.
本アプリケーションは,「特定の個人を見ていない場合,全員に小窓を表示する」という仕様になっている.
この仕様では,前回の人物クリックの結果,質問者側でプレゼンターの顔が正面に移動したのにもかかわらず,その上に小窓を表示され
2つの顔が映ってしまう.そのため,見られていない時も見られていると錯覚した可能性がある.

・OEBを使用する場合において,上記以外にも小窓の表示方法に関する意見が多くみられた.

・「顔の切り抜きはもう少し下に置くことは可能か?」

・「OEBについての説明が少ないので仕様がよく分からなかった。小窓がなくなった時がよくわからなかった。小窓が唐突に消えるのに驚いた。」

・よって,小窓の表示方法に関しては再考する必要がある.

・また,全天球ビデオカメラにおいて,被写体の顔が小さく表示されるという意見も多くみられた.

・「3人のうち誰か一人を見るということが難しいと感じた(視野に全員が収まっているので誰か一人にフォーカスして話しづらい)」

・「顔とカメラが遠い?からなのか、そもそも目が見えなかった」

・「発表者の顔の画像が暗くて小さいため、何となく正面を向いている時に目があっていそうだと感じた。曖昧であり、判断に困った。」(いずれも平面ディスプレイ使用時)

・全天周カメラ自体の視野の広さと,撮影領域上下が広がるような歪みによって,相対的に顔が小さくなることが原因であると考えられる.

・よって,全天周カメラを用いて顔を映したコミュニケーションを行う際には,何らかの方法で顔部分をクローズアップする必要がある.

・「ディスプレイのサイズ的に3人とも常に視野の中に収まってはいるので、クリックした人が真ん中に来る必要はないかな…と感じた。
画像そのものが移動するよりは、クリックしている間はカーソルの周辺に枠がでる…とかのほうが「ひとりにフォーカスしている感」があって良いかもしれないと思いました。」(OEB使用時,プレゼンター側)

・PCを使用するプレゼンター側において,常にカメラの方を向くことで見たい人間を正面に捉えるように設計したが,常に3人とも視界に入っているためかえって混乱を招くことになった.
