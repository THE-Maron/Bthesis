\begin{comment}
・表5.1,5.2を見るとOEB使用時,平面ディスプレイ使用時共に真陽性率(プレゼンターが見ていた時間の中で,目が合ったと感じた時間の割合)
の最大値,最小値の差が非常に大きい(極端な例は平面ディスプレイ使用時の0.00(一度もキーを押さなかった))

・自由記述意見の1つ「発表者の顔の画像が暗くて小さいため、何となく正面を向いている時に目があっていそうだと感じた。曖昧であり、判断に困った」(平面ディスプレイ使用時)
に見られるように,そもそも「目が合っている」という感覚自体があいまいで,人によって見られていると感じた機会が不ぞろいである原因であると考えられる.

・真陽性率は3人とも平面ディスプレイ使用時に対し,OEB使用後の方が増加している

・サンプル数の少なさ,及び順序効果により,比較はできないが,本実験で使用したアプリケーションによって
視線情報が共有されやすくなったという仮定に対し,一考の余地がある.

・一方で,視線情報を小窓の表示状態などからも判断できるOEB使用時の方が,偽陽性率が下がると予想していたが,
2者はOEB使用時の方が偽陽性率が増加した.特に,被検者2の偽陽性率は0.16上昇し0.28となっている.

・「正面に発表者がいて、さらに窓に発表者がいることに違和感を感じました。」(OEB使用時)という意見があった.
本アプリケーションは,「特定の個人を見ていない場合,全員に小窓を表示する」という仕様になっている.
この仕様では,前回の人物クリックの結果,質問者側でプレゼンターの顔が正面に移動したのにもかかわらず,その上に小窓を表示され
2つの顔が映ってしまう.そのため,見られていない時も見られていると錯覚した可能性がある.

・OEBを使用する場合において,上記以外にも小窓の表示方法に関する意見が多くみられた.

・「顔の切り抜きはもう少し下に置くことは可能か?」

・「OEBについての説明が少ないので仕様がよく分からなかった。小窓がなくなった時がよくわからなかった。小窓が唐突に消えるのに驚いた。」

・よって,小窓の表示方法に関しては再考する必要がある.

・また,全天球ビデオカメラにおいて,被写体の顔が小さく表示されるという意見も多くみられた.

・「3人のうち誰か一人を見るということが難しいと感じた(視野に全員が収まっているので誰か一人にフォーカスして話しづらい)」

・「顔とカメラが遠い?からなのか、そもそも目が見えなかった」

・「発表者の顔の画像が暗くて小さいため、何となく正面を向いている時に目があっていそうだと感じた。曖昧であり、判断に困った。」(いずれも平面ディスプレイ使用時)

・全天周カメラ自体の視野の広さと,撮影領域上下が広がるような歪みによって,相対的に顔が小さくなることが原因であると考えられる.

・よって,全天周カメラを用いて顔を映したコミュニケーションを行う際には,何らかの方法で顔部分をクローズアップする必要がある.

・「ディスプレイのサイズ的に3人とも常に視野の中に収まってはいるので、クリックした人が真ん中に来る必要はないかな…と感じた。
画像そのものが移動するよりは、クリックしている間はカーソルの周辺に枠がでる…とかのほうが「ひとりにフォーカスしている感」があって良いかもしれないと思いました。」(OEB使用時,プレゼンター側)

・PCを使用するプレゼンター側において,常にカメラの方を向くことで見たい人間を正面に捉えるように設計したが,常に3人とも視界に入っているためかえって混乱を招くことになった.
\end{comment}

\subsection*{定量的データの考察}

\subsubsection*{目が合うという感覚の曖昧さ}

\ref{rate1}と\ref{rate2}を見ると,OmniEyeBall使用時,平面ディスプレイ使用時共に
真陽性率の最大値と最小値の差が非常に大きいことが分かる.極端な例としては
平面ディスプレイ使用時に$0.00$という値が見られ,これは実験中に1度もキーを
押さなかったことを表している.

自由意見の一つに
\begin{itemize}
  \item 発表者の顔の画像が暗くて小さいため、何となく正面を向いている時に目があっていそうだと感じた。曖昧であり、判断に困った.(平面ディスプレイ使用時)
\end{itemize}
というものが見られた.今回,各個人の「目が合っている」という感覚に重点を置き,あえてそのままの形で
計測を行ったが,この結果にあらためて「目が合っている」と感じる状況そのものが
個人によって如何に異なるかが表れている.

\subsubsection*{真陽性率と偽陽性率について}

真陽性率については,3人とも平面ディスプレイ使用時に対し,
OmniEyeBall使用時の方が増加している.サンプル数の少なさ,
及び順序効果により,比較は出来ない.しかし,本実験で使用したアプリケーション
によって,視線情報が共有されやすくなったということは一考の余地がある.
\begin{itemize}
  \item 明示的に「話しかける対象を指定する」という動作をする必要があるので、モニターに複数人が映されているという話しかける相手を意識しづらい状況で、より今誰に話かけているかを強く意識できたのが良かった
\end{itemize}
というようにプレゼンターからのコメントも見られ,視線それ自体を意識する機会を向上させた可能性もある.

一方で,OmniEyeBall使用時の偽陽性率も,平面ディスプレイ使用時の
それ以上となっている.特に,被検者2の偽陽性率は0.16上昇し0.28となっている.
偽陽性率は,見られていないのに見られていると
錯覚した時間の割合であり,視線情報が正しく伝わっていないことを示す.
自身が見られている場合と見られていない場合で表示方法が異なる今回の仕様では
予期していない結果であった.この原因としては「特定の個人を見ていない場合,全員に小窓を表示する」
という仕様にあると考える.この仕様では,前回の人物クリックの結果,質問者側でプレゼンターの顔が正面に移動したのにもかかわらず,
その上に小窓を表示され2つの顔が映ってしまう.
全天球カメラの映像のままでは,見られていると錯覚するため,あえて小窓を上に被せた.
しかし,自由記述欄には以下のコメントが存在した.
\begin{itemize}
  \item 正面に発表者がいて、さらに窓に発表者がいることに違和感を感じました。(OmniEyeBall使用時)
\end{itemize}
加えて
\begin{itemize}
  \item OEBについての説明が少ないので仕様がよく分からなかった。小窓がなくなった時がよくわからなかった。小窓が唐突に消えるのに驚いた。(OmniEyeBall使用時)
\end{itemize}
このように,小窓の表示方法が複数存在し,かつ直感的でなかったことが原因であったと考えられる.

\subsection*{定性的データの考察}
5段階評価項目は,奇数番目の項目が肯定的な,偶数番目の項目が
否定的な項目となっている.サンプル数は4であり,比較は出来ない.
(実際に各質問事項に対し,対になっているデータのt検定を
有意水準$5\%$で行ったところ,p値の最小は
質問11の0.058であり,有意差は見られなかった.)

だが,質問事項4,11のように,回答に偏りがあるものも見られ,
これらから本アプリケーションの優れていた点と改良すべき点を模索する.

\subsubsection*{対面感についての考察}
質問11の結果に注目すると,平面ディスプレイの時は1又は2
であった回答がOmniEyeBall使用時には3又は4にまで上昇している.
これは,実験用アプリケーションを用いることで,ビデオ会議の参加者と
あたかも対面しているかのような感覚を増させることが出来たことを
示唆している.実際に,以下のようなコメントが見られた.
\begin{itemize}
  \item 発表者の見え方が対面の時に近くて、直感的であったため。また、質問者と発表者が会話している様子がわかりやすかったから。(記述式項目2の回答)
\end{itemize}

\subsubsection*{会話状況の把握についての考察}
5段階評価項目としては設けていないが,記述式項目2の回答に
会話状況が把握しやすくなったという旨の記述が4人中2人に見られた.
\begin{itemize}
  \item 発表者の見え方が対面の時に近くて、直感的であったため。また、質問者と発表者が会話している様子がわかりやすかったから。
  \item 今誰に対して話しているのかが明確になってわかりやすいから. 話聞いていなくて誰に対して話しているか分からない時に,即座に誰と話しているかどうかが認識できるので良いと思った.
\end{itemize}

\subsubsection*{アプリケーションの仕様の複雑さの考察}
質問4の結果を見ると,平均値に大きな差が出ており,OmniEyeBall
を使用するケースの方が否定的な結果になっている.質問4はアプリケーション
の仕様の分かりやすさについての項目である.自由記述のコメントには
本アプリケーションの仕様の複雑さの指摘が2つ見られた.
\begin{itemize}
  \item 正面に発表者がいて、さらに窓に発表者がいることに違和感を感じました。
  \item OEBについての説明が少ないので仕様がよく分からなかった。小窓がなくなった時がよくわからなかった。小窓が唐突に消えるのに驚いた。
\end{itemize}
どちらも小窓の表示方法に関する指摘である.
小窓の表示の改良は喫緊の課題であるといえる.

\subsubsection*{全天球カメラの視野に関する考察}
最後に,記述式項目の回答に,全天球カメラの映像
それ自体に関わるものがいくつか見られた.
\begin{itemize}
  \item 発表者の顔の画像が暗くて小さいため、何となく正面を向いている時に目があっていそうだと感じた。曖昧であり、判断に困った。(平面ディスプレイ使用時)
  \item 3人のうち誰か一人を見るということが難しいと感じた(視野に全員が収まっているので誰か一人にフォーカスして話しづらい(プレゼンター)
\end{itemize}
以上のように,ビデオ会議の参加者が小さく表示されてしまうという問題がある.
全天球ビデオカメラの視野の広さと相対的に,被写体が小さくなってしまったり,
あるいは正距円筒図法の特徴として,カメラ上・下部の映像ほど拡大されてしまうせいで,
中心付近に移る顔画像が相対的に小さくなってしまうことが原因として考えられる.

\subsection*{本節のまとめ}
以上の考察から,本システムの優れている可能性のある点として以下が挙げられる.
\begin{itemize}
  \item 視線の情報を伝えたり,得ようとする機会が向上する
  \item 立体的な映像から,対面しているかのような感覚を得やすい
  \item 会話状況の把握がしやすくなる
\end{itemize}

一方,改善すべき点としては以下が挙げられる.
\begin{itemize}
  \item 小窓の表示方法の改善
  \item PC側で表示する全天球パノラマ映像の顔部分の表示
\end{itemize}













