%・定量的な項目と定性的な項目それぞれ設けた

%・以下定量的項目

%・プレゼンター側が誰を見ていたのかをリアルタイムで記録

%・同時に,質問者側は,目が合っていると感じる間,キーを押しっぱなしにして時間を記録

%・見ているときに目が合ったと感じた割合(真陽性率)

%・見ていないのに目が合ったと感じた割合(偽陽性率)を算出

%・以下定性的項目

%・System Usability Scale に基づいた5段階評価項目10個

%・独自の評価項目(5段階評価2つ+記述式項目)

%・OEBを使用したケースとPCを使用したケース両方を経験した被検者に対し,どちらの方が好ましいかを問う
%記述式項目を設けた

実験では,定量的な評価と定性的な評価をそれぞれ設けた.定量的評価の項目は
以下の通り.

\begin{itemize}
  \item プレゼンターが誰を見ていたのかをリアルタイムで記録(OmniEyeBall使用時は人物のクリック,
  平面ディスプレイ使用時は人物を実際に見てもらうことが,見るという動作に相当する)前者はアプリケーションの
  操作ログをTkinterで取得することで記録した.後者はキーボードの指定のキーを押し続けることで記録した.
  \item 同時に,傍聴者側はプレゼンターと目が合っていると感じる間,キーボードの指定のキーを押しっぱなしに
  して,その操作を記録する.
  \item 記録から,見ているときに目が合ったと感じた割合(真陽性率)及び,見ていないのに目が合ったと感じた割合
  (偽陽性率)を算出した.
\end{itemize}

また,定性的な評価は,被検者アンケートにより行った.
アンケートの内容は
SUS(System Usability Scale)に基づいた5段階評価項目10個.
また独自の評価項目5段階評価項目2つ,記述式項目を1つ設けた.(表\ref{question})
質問11は,遠隔会議の臨場感を問うものであり,
球場ディスプレイを使用したことや,本研究で
実装した視線情報を伝える機能が,臨場感の向上に
貢献していたかどうかを調べるために設けた.
また,質問12は,各状況においての対面時との差異が
いかほどかを調べるために設けた質問である.
さらに,平面ディスプレイを使用するケースとOmniEyeBallを使用する
ケース双方が修了した後は,どちらの方が好ましいか理由付きで問う
記述式項目を設けた.

\begin{table}[tp]
  \begin{center}
  \begin{tabular}{|c|c|}
  \hline
  質問1   & このアプリケーションはしょっちゅう使いたくなるだろうと感じた.     \\ \hline
  質問2   & このアプリケーションは必要以上に複雑だと感じた.            \\ \hline
  質問3   & このアプリケーションは使いやすいと感じた.               \\ \hline
  質問4   & このアプリケーションを使用するのに専門家のサポートが必要だと感じた.  \\ \hline
  質問5   & このアプリケーションの機能はうまくまとまっていると感じた.       \\ \hline
  質問6   & このアプリケーションにはちぐはぐな部分が多くあると感じた.       \\ \hline
  質問7   & このアプリケーションの使い方は大抵の人がすぐに理解するだろうと感じた. \\ \hline
  質問8   & このアプリケーションはとても扱いづらいと感じた.            \\ \hline
  質問9   & このアプリケーションを使いこなせると確信している.           \\ \hline
  質問10  & このアプリケーションを利用するまでに学ぶことが多いと感じた.      \\ \hline
  質問11  & 対面のプレゼンテーションのように目の前に通話相手がいるように感じた.  \\ \hline
  質問12  & 対面のプレゼンテーションにはない違和感を多く感じた.          \\ \hline
  記述式項目1 & (質問者側の場合)どのような時に目が合ったと感じたかお答えください.  \\ \hline
  記述式項目2 & OEB使用、平面ディスプレイ使用、どちらのケースが好きだったかお答えください. \\ \hline
  \end{tabular}
  \caption{質問項目の一覧}\label{question}
  \end{center}
  \end{table}
