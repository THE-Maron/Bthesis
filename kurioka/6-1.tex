\begin{comment}
・また,SUSのネガティブな評価項目がOEB使用時で有意差は見られないものも,5に近くなってしまっている.

・OEB使用時のアンケート結果でGUIに関する意見が多く出ている.

・特に「仕様が分かりづらい」,「小窓の表示位置を変えたい」という趣旨の意見が多くみられた.

・小窓の位置のみならず,本来の全天球カメラの映像に被さって小窓が表示される仕様など,映像が複雑になってしまう状況もあり,
それが混乱を招いた可能性がある.

・また,全天球カメラ自体の視野の広さから,パノラマ映像では顔が小さく表示されやすく,PCの画面に表示する
全天周カメラの映像表示方法にも改善の余地がある.

・算出した偽陽性率に改善が見られなかった点から,小窓に表示する顔の表示方法自体も変更する余地がある.

・実験結果に関係しない事項としては,実験用アプリでは被験者の位置が変更しない制限があったが,
実際は被写体は自由に動くことが出来るべきである.被写体が移動しても正しく小窓を表示できるように
する必要がある.
\end{comment}
5-4-3節で,立体的な映像から,対面しているかのような感覚を得やすいという結論
が得られた.これに基づいて,実際に対面した時に見えるような立体視を
再現することが効果的であると考えられる.

一方で,5段階評価項目の結果では,OmniEyeBall使用時の方が
悪い結果が出ている項目も多く,実験用アプリケーションには
使用感に問題があったと考えられる.自由記述アンケートでは
「仕様が分かりづらい」,「小窓の表示位置を変えたい」という趣旨の意見が多くみられたことに
裏付けられている.実験用アプリケーションでは,本来の全天球カメラの映像に被さって小窓が表示される仕様など,
映像が複雑になってしまう状況もあり,
それが混乱を招いた可能性がある.
ままた,全天球カメラ自体の視野の広さから,パノラマ映像では顔が小さく表示されやすく,PCの画面に表示する
全天周カメラの映像表示方法にも改善の余地がある.

実験結果に関係しない事項としては,実験用アプリでは被験者の位置が変更しない制限があったが,
実際は被写体は自由に動くことが出来るべきである.被写体が移動しても正しく小窓を表示できるように
する必要がある.

以上の点を踏まえ,改善すべき点を反映させたアプリケーションを
以降で提案する.