\begin{comment}
・また,SUSのネガティブな評価項目がOEB使用時で有意差は見られないものも,5に近くなってしまっている.

・OEB使用時のアンケート結果でGUIに関する意見が多く出ている.

・特に「仕様が分かりづらい」,「小窓の表示位置を変えたい」という趣旨の意見が多くみられた.

・小窓の位置のみならず,本来の全天球カメラの映像に被さって小窓が表示される仕様など,映像が複雑になってしまう状況もあり,
それが混乱を招いた可能性がある.

・また,全天球カメラ自体の視野の広さから,パノラマ映像では顔が小さく表示されやすく,PCの画面に表示する
全天周カメラの映像表示方法にも改善の余地がある.

・算出した偽陽性率に改善が見られなかった点から,小窓に表示する顔の表示方法自体も変更する余地がある.

・実験結果に関係しない事項としては,実験用アプリでは被験者の位置が変更しない制限があったが,
実際は被写体は自由に動くことが出来るべきである.被写体が移動しても正しく小窓を表示できるように
する必要がある.
\end{comment}
5章で,立体的な映像から,対面しているかのような感覚を得やすいという結論
が得られた.これに基づいて,実際に対面した時に見えるような立体視を
再現することが効果的であると考えられる.6章ではより立体感のある
小窓の表示方法や,全天球カメラの映像の表示方法を組み込んだアプリケーションを
提案する.

加えて,小窓の表示位置の改善を行う.そうすることで,よりユーザーに対して
分かりやすい映像の表示方法を提案する.

実験結果に関与しないが,改善すべき点として,実験と違い
被写体が移動するという条件が加わる.被写体が移動しても
動的に小窓の表示位置を変更できるように,顔認識ライブラリを
用いた手法を組み込む.
