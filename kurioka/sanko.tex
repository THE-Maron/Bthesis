
\bibitem{1}zoomの説明
\bibitem{2}google meetの説明
\bibitem{3}The GAZE groupware system: mediating joint attention in multiparty communication and collaborationの文献
\bibitem{4}Theta Vの説明(https://theta360.com/ja/about/theta/)
\bibitem{5}Collaboration in 360° Videochat: Challenges and Opportunitiesの文献
\bibitem{6}Can You See Me Now?: How Field of View Affects Collaboration in Robotic Telepresenceの文献
\bibitem{7}OBS-virtual-camの説明
\bibitem{8}UnityCaptureの説明(https://github.com/schellingb/UnityCapture)
\bibitem{9}OEBStudioの説明
\bibitem{10}おかしら会議の文献
\bibitem{11}Pose-Guided Photorealistic Face Rotationの文献
\bibitem{12}Dlibの説明(http://dlib.net/)
\bibitem{13}https://ci.nii.ac.jp/naid/40021907210
\bibitem{14}%https://www.jstage.jst.go.jp/article/hozen/advpub/0/advpub_1822/_pdf/-char/ja
\bibitem{15}glomal 350の説明(https://www.aisan.co.jp/products/glomal350.html)
\bibitem{16}Unity https://unity.com/ja
\bibitem{17}有志によるOpenCVの解説(http://labs.eecs.tottori-u.ac.jp/sd/Member/oyamada/OpenCV/html/py\_tutorials/py\_feature2d/py\_matcher/py\_matcher.html)
\bibitem{18}OmniEyeBall(https://dl.acm.org/doi/10.1145/3266037.3266092)
\bibitem{19}Collaboration in 360° Videochat: Challenges and Opportunities
\bibitem{20}%https://ja.wikipedia.org/wiki/%E3%83%86%E3%82%A4%E3%82%BD%E3%83%BC%E3%81%AE%E6%8C%87%E7%A4%BA%E6%A5%95%E5%86%86#/media/%E3%83%95%E3%82%A1%E3%82%A4%E3%83%AB:Tissot_indicatrix_world_map_equirectangular_proj.svg
\bibitem{21}Meeting OWL(http://meetingowl.jp/?i=nav)
\bibitem{22}WORLDEYE
\bibitem{23}iSphere: Self-Luminous Spherical Drone Display
\bibitem{24}How Display Shapes Affect 360-Degree Panoramic Video Communication
\bibitem{25}An Interactive Omnidirectional Ball Display
\bibitem{26}Comparing flat and spherical displays in a trust scenario in avatar-mediated interaction
\bibitem{27}Room2Room: Enabling Life-Size Telepresence in a Projected Augmented Reality Environment
\bibitem{28}Improving Visibility of Remote Gestures in Distributed Tabletop Collaboration
\bibitem{29}A Gaze-preserving Group Video Conference System using Screen-embedded Cameras
(後で正式な書式に書き直します)
