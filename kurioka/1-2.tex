本研究では,個人対複数人というシチュエーションの
遠隔会議において,臨場感を高めることを目的とする.
この形態の遠隔会議は,従来から行われてきたものの,
コロナ禍の影響において,今後さらに増えていくと予想される.

この際,参加者は全天球カメラを使用し,個人側はPCに,
複数人側はOmniEyeBallをディスプレイとして使用する,
非対称なビデオコミュニケーションを行うものとした.
OmniEyeBallの使用理由は,立体的な表示で
臨場感を高めることである.一方,個人のPCの利用理由は,
入手コストの観点から,個人での利用機会が多いものを選択した.

1-1節で述べたように,全天球カメラの使用者がどこを見ているか
知る手段が必要である.特に,今回の状況では,個人側のディスプレイには
複数人が表示され,通信相手に誰を見ているのか伝えるのが重要である.
その方法として,映像の回転を行う.そうすることで,PC使用者が見ている人物の
方向に対して,映像に表示されたPC使用者の顔が,見られている人物の
方向を向くようにする.視線情報が伝わると同時に,顔が向くといった
対面時での現象を再現することによって,会話の没入感を高めること
も目指した.

上記の操作を実装したシステムを用いて,実験を行い,
正確に視線情報が伝わるかを調べる.また,使用感に
ついてのアンケートを行い,臨場感が上昇したか,及び
その理由を調査する.その結果から臨場感を高める要素
が何であるのかを見出し,それを反映したいくつかのアプリケーション
を提案する.