%・RICHO THETA V

%・RICHO THETA S

%・(OmniEyeBall)4K画質球体ディスプレイを使用

%・glomal360

\subsection*{全天周ビデオカメラ}
全天周ビデオカメラとして,PC側ではRICOH THETA S,
OmniEyeBall側ではRICOH THETA Vを用いた.
\begin{table}[tbp]
  \begin{center}
    \begin{tabular}{|c|c|}
      \hline
      記録媒体                                                             & 内蔵メモリー:約8GB                                                                                                                                                    \\ \hline
      記録可能枚数, 時間                                                       & \begin{tabular}[c]{@{}l@{}}静止画:(L)約1600 枚, (M)9000 枚\\ 動画(1 回の記録時間)\\ :最大25 分もしくはファイルサイズの上限4GB\\ 動画(合計記録時間)\\ :(L) 約65 分, (M) 約175 分\end{tabular}              \\ \hline
      圧縮方式                                                             & \begin{tabular}[c]{@{}l@{}}静止画:JPEG(Exif Ver2.3) DCF2.0 準拠\\ 動画:MP4(映像:MPEG-4 AVC/H.264, 音声:AAC)\\ ライブストリーミング\\ :(USB)MotionJPEG, MPEG-4 AVC/H264\end{tabular} \\ \hline
      静止画解像度                                                           & \begin{tabular}[c]{@{}l@{}}L:5376 × 2688\\ M:2048 × 1024\end{tabular}                                                                                          \\ \hline
      \begin{tabular}[c]{@{}l@{}}動画解像度/フレームレート/\\ ビットレート\end{tabular}  & \begin{tabular}[c]{@{}l@{}}L:1920 × 1080/30fps/16Mbps\\ M:1280 × 720/15fps/6Mbps\end{tabular}                                                                  \\ \hline
      \begin{tabular}[c]{@{}l@{}}ライブストリーミング解像度/\\ フレームレート\end{tabular} & \begin{tabular}[c]{@{}l@{}}L:1920 × 1080/30fps\\ M:1280 × 720/15fps\end{tabular}                                                                               \\ \hline
      \end{tabular}
  \caption{THETA Sの仕様} \cite{4}
  
  \begin{tabular}{|c|c|}
    \hline
    記録媒体                                                             & 内蔵メモリー:約19GB                                                                                                                                                                    \\ \hline
    記録可能枚数, 時間                                                       & \begin{tabular}[c]{@{}l@{}}静止画:約4800枚\\ 動画(1 回の記録時間)\\ :最大25 分\\ 動画(合計記録時間)\\ :130分(2K, H264)\end{tabular}                                                                      \\ \hline
    圧縮方式                                                             & \begin{tabular}[c]{@{}l@{}}静止画:JPEG(Exif Ver2.3)\\ 動画:MP4(映像:MPEG4 AVC/H.264,H.265 *7、\\ 音声:AAC-LC(モノラル)+Linear PCM(4ch空間音声))\\ ライブストリーミング:(映像:H.264、音声:AAC-LC(モノラル))\end{tabular} \\ \hline
    静止画解像度                                                           & 5376×2688                                                                                                                                                                       \\ \hline
    \begin{tabular}[c]{@{}l@{}}動画解像度/フレームレート/\\ ビットレート\end{tabular}  & \begin{tabular}[c]{@{}l@{}}4K,H264:3840×1920/29.97fps/56Mbps\\ 2K,H264:1920×960/29.97fps/16Mbps\end{tabular}                                                                    \\ \hline
    \begin{tabular}[c]{@{}l@{}}ライブストリーミング解像度/\\ フレームレート\end{tabular} & \begin{tabular}[c]{@{}l@{}}4K,H264:3840×1920/29.97fps/120Mbps\\ 2K,H264:1920×960/29.97fps/42Mbps\end{tabular}                                                                   \\ \hline
    \end{tabular}
    \caption{THETA Vの仕様} \cite{4}
  \end{center}
\end{table}

\subsection*{球体ディスプレイ}
全天球動画の出力装置としては
渋谷光学の球体プロジェクターであるGlomal350を使用する.
Glomal350及び,THETA Vを利用して,OmniEyeBallを設計した.()
下部に設置した上向きのプロジェクターにGlomal 350を取付,
上部に設置した球体カプセルに内部から全天球動画を投影する.
カプセルの天頂部分にはTHETA Vを取り付けるスタンドを設置し,
THETA Vを取り付けることで,ディスプレイ周囲の全天球映像を
撮影した.THETA Vとプロジェクターは,それぞれUSBとHDMIで
Mac book proと接続されており,THETA Vの映像を送信し,受信した映像は
プロジェクターから投影する.

\begin{table}[tbp]
  \begin{center}
    \begin{tabular}{|c|c|}
      \hline
      外形寸法    & \begin{tabular}[c]{@{}l@{}}幅450X奥行460X高さ600mm\\      球体サイズ:φ350mm\end{tabular} \\ \hline
      投影方式    & 1clip DLP方式                                                                    \\ \hline
      表示素子サイズ & XGA0.55型(アスペクト比4:3)                                                            \\ \hline
      画素数     & 786,432画素(1024X768)                                                            \\ \hline
      明るさ     & 2700lm                                                                         \\ \hline
      HDMI信号  & \begin{tabular}[c]{@{}l@{}}圧縮表示:\\      最大1920X1080(HD TV 1080p)\end{tabular}  \\ \hline
      \end{tabular}
  \caption{Glomal 350の仕様} \cite{15}
  \end{center}
  \end{table}

  \begin{figure}[tbp]
    \centering
    \includegraphics[scale=0.16]{fig/OEBnow.png}
    \caption{今回の実験で用いたOmniEyeBall}
  \end{figure}

  \begin{figure}[p]
    \centering
    \includegraphics[scale=0.16]{fig/glomal350_now.png}
    \caption{プロジェクターのレンズに取り付け固定したGlomal 350}
  \end{figure}
