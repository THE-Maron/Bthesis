%・全天球映像は視野が広い

%・ビデオ会議では視線情報が失われる

%・誰がどこを見ているのかという情報が必要

%・(上記の根拠として,2で関連研究を提示する)

ビデオ会議では,視野の狭さにより,参加者同士の状況が分かりづらいという欠点がある.
そのため,コミュニケーションが円滑に行われない場合がある.全天球ビデオカメラを使用することで
視野の狭さの問題が解決するが,一方で映像が複雑になり,その影響でやはり快適なコミュニケーションが
行われない可能性がある.

一方で,2-2節で述べたOmniEyeBallに代表される,全天周ディスプレイを用いたビデオ会議が考えられる.
全天周ディスプレイでは,全天球映像が球を用いた立体的かつ自然な形で表現され,全天球映像を視認
しやすくなる.

しかし,上記の方法でなお,視線情報の問題が存在する.これは,カメラ画像から発話者の視線が取得困難であったり,
発話者の視線が考慮されたディスプレイ上での表示が行われていない,といった
コミュニケーションにおいて重要であるにもかかわらず従来のビデオ会議では未だに未解決の問題である.
以下では,PCの使用者1人とOmniEyeBallの使用者複数人の非対称な通話状況において,PC使用者の視線情報を伝える方法を提案する.



