%・全天球映像は視野が広い

%・ビデオ会議では視線情報が失われる

%・誰がどこを見ているのかという情報が必要

%・(上記の根拠として,2で関連研究を提示する)

ビデオ会議では,視野の狭さにより,参加者同士の状況が分かりづらいという欠点がある.
そのため,コミュニケーションが円滑に行われない場合がある.全天球ビデオカメラを使用することで
視野の狭さの問題が解決するが,一方で映像が複雑になり,その影響でやはり快適なコミュニケーションが
行われない可能性がある.

一方で,先ほど述べたOmniEyeBallに代表される,全天周ディスプレイを用いたビデオ会議が考えられる.
全天周ディスプレイでは,全天球映像が球を用いた立体的かつ自然な形で表現され,全天球映像を視認
しやすくなる.

しかし,上記の方法でなお,視線情報の問題が存在する.従来のビデオ会議では,参加者がどの参加者を見ているかなどの
視線情報が伝わらない.以下では,PCの使用者1人とOmniEyeBallの使用者複数人の非対称な通話状況において,
PC使用者の視線情報を伝える方法を提案する.以下で提案したシステムを用いて,従来のビデオ通話アプリケーション
を用いた場合よりもPCの使用者の視線情報がOmniEyeBallの使用者に正確に伝わることが最終的な目標である.



