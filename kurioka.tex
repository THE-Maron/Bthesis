\documentclass[a4j,11pt,oneside,openany,report]{jsbook}

\usepackage{comment}
\usepackage[a4paper,truedimen,margin=25truemm]{geometry}
\usepackage{cscover}
\usepackage[dvipdfmx]{graphicx}
\usepackage[nobreak]{cite}
\usepackage[a4paper,dvipdfmx,pdfdisplaydoctitle=true,%
    bookmarks=true,bookmarksnumbered=true,bookmarkstype=toc,bookmarksopen=true,%
    pdftitle={全天周ビデオ通信端末を用いた非対称型コミュニケーションに関する研究},%
    pdfauthor={栗岡 保}%
    ]{hyperref}
\usepackage{pxjahyper}

\renewcommand{\bibname}{参考文献}
\setcounter{tocdepth}{2}
\pagestyle{plain}

\newcommand{\TODO}[1]{\textbf{[TODO: #1]}}
%\renewcommand{\TODO}[1]{}

\thesistype{学士特定課題研究論文}
\title{全天周ビデオ通信端末を用いた非対称型コミュニケーションに関する研究}
\author{栗岡 保}
\studentid{17B05518}
\affiliation{東京工業大学\\情報理工学院\\情報工学系} 
\date{2021年1月}

\supervisorname{指導教員}
\supervisor{小池 英樹}
%\dsupervisorname{副指導教員}
%\dsupervisor{工学 次郎}

\begin{document}

\frontmatter
\maketitle

\chapter{概要}
学士特定課題研究論文は、シングルカラムでページ数に制限はない。

\tableofcontents
\listoffigures
\listoftables

%%%%%%%%%%%%%%%%%%%%%%%%%%%%%%%%%%%%%%%%%%%%%%%%%%

\mainmatter
\chapter{序論}

\section{本章の概要} 本章では、本研究の背景、目的、本論文の構成について述べる。
\section{本研究の背景}近年、様々なビデオ会議アプリケーション(図\ref{fig:1})が登場している。
その例としては、zoom\cite{1}やGoogle Meet\cite{2}が挙げられる。
新型コロナウイルス感染症の流行により、ビデオ会議の需要がさらに高まりつつある。
\begin{figure}[tb]
  \centering
  \includegraphics[scale=0.5]{fig/zoom-monitor-screen.png}
  \caption{zoom}\label{fig:1}\cite{1}
\end{figure}

しかし、ビデオ会議における様々な問題点も指摘されている。
例えばRoel\cite{3}は、カメラの視覚外の情報や、人物の情報の不足のために、対面時のような
インタラクションを得られないことを指摘している。

一方で、昨今は誰でも気軽に全天球映像(図\ref{fig:2})を撮影することができるようになっている。
その例として、全天球カメラ(360度カメラ,全天周カメラ,全方位カメラなどともいう (図\ref{fig:3}))
を使用して、全天球映像をヘッドマウントディスプレイを用いて観覧したり、パノラマ映像として動画や静止画を
保存することが出来るようになっている。
\begin{figure}[tb]
  \centering
  \includegraphics[scale=0.2]{fig/thetaV.png}
  \caption{全天球パノラマ画像}\label{fig:2}
\end{figure}
\begin{figure}[tb]
  \centering
  \includegraphics[scale=0.2]{fig/thetaV.png}
  \caption{theta V}\label{fig:3}\cite{4}
\end{figure}

全天周カメラの使用により、カメラの視野の問題は解決される。
実際に、Anthonyら\cite{5}は、全天球の視野の広さによって、リモートユーザーが
ローカルユーザーの環境をより早く理解できると結論付けている。
だが、Johnsonら\cite{6}によって、パノラマ視野によって映像の複雑さが増し、
より大きな認知負荷を必要としたことも示されている。

認知負荷を削減するため、パノラマ映像を3次元に表示する、球体プロジェクターを
用いた方法が挙げられる。球体プロジェクターの利用としては、LiらのOmniEyeBall
\section{本研究の目的}
\section{本論文の構成}

\chapter{関連研究}
%\section{本章の概要}
\section{360度カメラに関する研究}\begin{comment}
\begin{itemize}
  \item RICOH THETA V(https://theta360.com/ja/about/theta/v.html)
  \item insta 360(https://www.insta360.com/jp/product/insta360-onex2)
  \item Meeting OWL(http://meetingowl.jp/?i=nav)
  \item Collaboration in 360° Videochat: Challenges and Opportunities(https://dl.acm.org/doi/10.1145/3064663.3064707)
\end{itemize}
\end{comment}

\subsubsection*{全天球カメラ}

全天球カメラの具体例としては,RICHOのTHETA V\cite{4}が挙げられる.

THETAの裏と表には,それぞれ1枚ずつ,計2枚の魚眼レンズが
取り付けられている.本体を薄く設計することで,2つの
魚眼レンズの視差を小さくして,後述するスティッチング処理を行った際
違和感のない画像を生成することを可能にしている.

TEHTAのパノラマ画像は,以下のようにして生成される.
まず,2枚の魚眼レンズで撮影された映像それぞれについて,
通常のカメラでも行われるような基本的な画像処理と,
映像それぞれの明るさや色味が合うようにする処理を行う.
生成された2枚の画像に対し,パターンマッチングを行い,
つなぐ位置を検出する.一方で,各魚眼レンズには
特有の歪みが存在し,Thetaではその歪みを補正する処理を
行っている.この2つの操作(つなぎ処理スティッチングと呼ぶ)を行い,
2枚の画像を結合することによって全天球パノラマ映像を生成している.
(図\ref{theta1},\ref{theta2}参照)

\begin{figure}[tp]
  \centering
  \includegraphics[scale=0.6]{fig/thetaref1.jpg}
  \caption{画像処理の流れ\cite{4}}\label{theta1}
\end{figure}

\begin{figure}[tp]
  \centering
  \includegraphics[scale=0.6]{fig/thetaref2.jpg}
  \caption{つなぎ処理スティッチング\cite{4}}\label{theta2}
\end{figure}

一方でパノラマ画像を生成する際に用いられる正距円筒図法であるが,
これには欠点が存在する.正距円筒図法は,球面を平面に投影する
方法の一種であり,緯線と経線が直行し,それぞれが等間隔になるように
表示された図法であり,世界地図などに広く用いられてきた.

投影の方法上,どの経線の長さも赤道の長さと一致するようになっているが
これは球面上の経線の性質と一致していない.実際,球面上では

\begin{eqnarray}
  (緯度\theta の経線の長さ):(赤道の長さ)= \cos{\theta}:1 \nonumber \\
\end{eqnarray}

であり,正距円筒図法上では,緯度$\theta$部分が横方向に$\frac{1}{cos\theta}$だけ
拡大されていることになる.この拡大によって生じた歪みを
わかりやすく示したものがテイソーの指示楕円である.(図\ref{teiso})
高緯度領域ほど,横方向に拡大されて楕円が大きくなる様子が図示されている.

\begin{figure}[tp]
  \centering
  \includegraphics[scale=1.0]{fig/teiso1.png}
  \includegraphics[scale=0.2]{fig/teiso2.png}
  \caption{テイソーの指示楕円\cite{20}}\label{teiso}
\end{figure}

全天球カメラを利用し,さらにパノラマ画像をそのまま表示するのではなく,
臨場感のあるビデオ会議を可能にしたデバイスとしてはMeetingOWL\cite{21}がある.

\begin{figure}[tp]
  \centering
  \includegraphics[scale=0.7]{fig/OWL.png}
  \caption{MeetingOWLの画面\cite{21}}
\end{figure}

\subsection*{全天球カメラの応用}

MeetingOWLは,マイクによる声の入力と,全天球映像から得られる話者の
姿勢によって,現在会話中の人物に焦点を当て,拡大して表示する
システムである.位置関係を損なわない画像の拡大により視認性が向上し,
臨場感が増している.しかし,通信相手が映像に映っている人物の誰を見ている
かについては,依然共有方法が存在していない.

全天球カメラを使用する効果に関する研究は,Tangら\cite{19}が行っている.
Tangらは,360度映像ビデオチャットアプリを作成し,それを利用して
遠隔地から現地の作業相手を支援するタスクの実験を行った.その実験の結果,
全天球カメラを用いることの利点として,以下を挙げている.

\begin{itemize}
  \item 全天球カメラの視野の広さのおかげで,カメラ位置の調整の必要が少なくなり,
  遠隔地の参加者が自由に見たい場所を見ることができるようになった.
  \item 視野の広さの影響で,遠隔地の参加者がより多くの情報を獲得でき,より
  良く作業を支援することができた.
\end{itemize}

全天球カメラは,使用者の周囲全体を映しているので,遠隔地の参加者は使用者の
周辺を自由に見渡せ,使用者の状況をすぐに知ることが出来る.一方で,全天球カメラ
の問題点として,以下を挙げている.

\begin{itemize}
  \item 遠隔地の参加者は,自由に映像を見ることが出来る一方で,現地の参加者
  と見ているものが一致しない場合があった.
  \item カメラの角度調節などを指示する際の語彙が不足していた.(カメラを上方向に
  傾ける操作を「上げる」と呼び,それがカメラを上昇させる操作と誤解される等)
\end{itemize}

ここに挙げた問題点の1つ目は,全天球カメラの視野の広さに起因するものであり,2つ目は
コミュニケーション自体が複雑化したことに起因するものである.視線情報の
共有や,3次元的な複雑な情報の直観的な表示方法を考えることが重要である.
\section{全天周球状ディスプレイに関する研究}\begin{itemize}
  \item Glomal350(https://www.aisan.co.jp/products/glomal350.html)
  \item WORLDEYE(https://www.gakkensf.co.jp/worldeye/sp/)
  \item OmniEyeball: An Interactive I/O Device For 360-Degree Video Communication(https://dl.acm.org/doi/10.1145/3279778.3279926)
  \item Qoom: An Interactive Omnidirectional Ball Display(https://dl.acm.org/doi/10.1145/3126594.3126607)
  \item Comparing flat and spherical displays in a trust scenario in avatar-mediated interaction(https://dl.acm.org/doi/10.1145/2556288.2557276)
\end{itemize}
\section{ビデオ会議に関する研究}\begin{comment}
\begin{itemize}
  \item Room2Room: Enabling Life-Size Telepresence in a Projected Augmented Reality Environment(https://dl.acm.org/doi/pdf/10.1145/2818048.2819965)
  \item Improving visibility of remote gestures in distributed tabletop collaboration(https://dl.acm.org/doi/10.1145/1958824.1958839)
  \item Putting things in focus: establishing co-orientation through video in context(https://dl.acm.org/doi/10.1145/2470654.2466174)
  \item A gaze-preserving group video conference system using screen-embedded cameras(https://dl.acm.org/doi/10.1145/3139131.3141775)
  \item Towards Enabling Eye Contact and Perspective Control in Video Conference (https://dl.acm.org/doi/10.1145/3379350.3416197)
\end{itemize}
\end{comment}
Pejsaら\cite{27}は,投影型の拡張現実環境を用いて,対面時の
ような1対1の遠隔会議が出来るシステムを考案した.Kinnectによって
参加者の動きや体勢を認識して,3Dモデルを自動生成して,相手の
正面へと表示することで,疑似的な対面環境を作り出した.
実験の結果,従来のビデオ会議を用いたコミュニケーションよりも
通信相手の存在感や,コミュニケーションの効率といった観点で
優れていることが示された.一方で,解像度の低さなどの問題から
,ジェスチャーを用いたコミュニケーションの頻度が低下したことや,
1対1のケースに用途が限られている等の問題点が指摘されている.
\begin{figure}[tp]
  \centering
  \includegraphics[scale=1.2]{fig/2room2.png}
  \caption{Room2Roomによる通信の様子\cite{27}}
\end{figure}

Yamashitaら\cite{28}は,卓上での作業における遠隔ビデオ会議
において,ジェスチャーの情報が失われることを指摘し,卓上
において遠隔参加者のジャスチャーが視認できるシステムt-Roomを
提案した.卓上ディスプレイに遠隔地参加者のジェスチャーを表示する
際には,物体にさえぎられるなどの問題がある.そこで,リモートラグと呼ばれる
,遠隔地参加者の遅延した映像をリアルタイム映像と同時に表示することで,
ジェスチャーの理解の正誤判断を行えるようにした.
通常表示条件とリモートラグ表示条件での比較実験の結果,ユーザーは
見落としたジェスチャーの情報などを後から再認識することに成功し,ジェスチャー
を見落とすことによる不要な会話が減少したことが示されている.また,作業参加者の
体感作業負荷を減少させたことも示されている.

\begin{figure}[tp]
  \centering
  \includegraphics[scale=0.7]{fig/tRoom.png}
  \caption{t-Roomを使用する様子\cite{27}}
\end{figure}

\begin{figure}[tp]
  \centering
  \includegraphics[scale=1.0]{fig/2room2.png}
  \caption{t-Roomの構成\cite{27}}
\end{figure}

Kobayashiら\cite{29}は,Kinnectと,複数のカメラから視線方向の推定を行い,
リアルタイムで,対面時の会話と同じ景色を再現する映像表示方法を提案した.(図\ref{Kobayashi})
Kobayashiらは,2対2でのコミュニケーションを行うケースを想定しており,以下の
場合について表示方法の切り替えを行った.
\begin{itemize}
  \item 一方の側の人物が他方の側のユーザに見られていない場合は,デフォルトの視点位置を使用する.
  \item  一方の面の人物が他方のユーザからのみ見られている場合は,そのユーザの視点位置を使用する.
  \item  一方の面の人物が他方の面の複数のユーザから見られている場合は,お互いを見ているペアを使用する.
\end{itemize}

\begin{figure}[tp]
  \centering
  \includegraphics[scale=0.8]{fig/gaze.png}
  \caption{Kobayashiらの提案\cite{29}}\label{Kobayashi}
\end{figure}

\begin{figure}[tp]
  \centering
  \includegraphics[scale=0.8]{fig/kobayashi2.png}
  \caption{Kobayashiらの提案システムのデモンストレーション\cite{29}}
\end{figure}




\section{まとめ}

\chapter{球体ディスプレイを用いた視線共有システムの提案}
%\section{本章の概要}
\section{従来型ビデオ会議アプリケーションの問題点}%・全天球映像は視野が広い

%・ビデオ会議では視線情報が失われる

%・誰がどこを見ているのかという情報が必要

%・(上記の根拠として,2で関連研究を提示する)

ビデオ会議では,視野の狭さにより,参加者同士の状況が分かりづらいという欠点がある.
そのため,コミュニケーションが円滑に行われない場合がある.全天球ビデオカメラを使用することで
視野の狭さの問題が解決するが,一方で映像が複雑になり,その影響でやはり快適なコミュニケーションが
行われない可能性がある.

一方で,2-2節で述べたOmniEyeBallに代表される,全天周ディスプレイを用いたビデオ会議が考えられる.
全天周ディスプレイでは,全天球映像が球を用いた立体的かつ自然な形で表現され,全天球映像を視認
しやすくなる.

しかし,上記の方法でなお,視線情報の問題が存在する.これは,カメラ画像から発話者の視線が取得困難であったり,
発話者の視線が考慮されたディスプレイ上での表示が行われていない,といった
コミュニケーションにおいて重要であるにもかかわらず従来のビデオ会議では未だに未解決の問題である.
以下では,PCの使用者1人とOmniEyeBallの使用者複数人の非対称な通話状況において,PC使用者の視線情報を伝える方法を提案する.




\section{提案する視線共有システム}%・(PC側では)クリックした人間の映像を正面に持ってくることで,正面を見ていれば見たい人を物理的にも見ているという状態を作る

%・(OEB側では)PC側が見ている特定の個人の方向に,PC使用者の顔を映す

%・残りの人間には,PC側と,見ている人間の顔を小さいウインドウに移し,視線情報と通信相手の顔を伝える

%・特定の1人を見ていない状況を区別するため,そのような状況下では全員にPC使用者の顔を小さいウインドウに移す

PC側で提案するシステムを以下に述べる.平面ディスプレイ上に
OmniEyeBallに取り付けた全天球ビデオカメラのパノラマ映像を表示する.
映像上で,参加者の顔をクリックすることで,その参加者が正面に来るように
パノラマ映像を平行移動させる.そのようにして,常に正面を見ていれば
見たい人を見ているという状態を作り出す.

%(ここに,画像をスライドさせる図を乗せる)
\begin{figure}[tbp]
  \centering
  \includegraphics[scale=0.6]{fig/PCsideSlide.png}
  \caption{全天周パノラマ画像のスライド}
\end{figure}

以降は,OmniEyeBall側で提案するシステムを述べる.
PC側で使用する全天周ビデオカメラの映像を
球体ディスプレイ上に表示する.PC側のクリック操作に合わせ
クリックされた人物の方向に顔が向くように映像を回転させる.
こうすることで,PC側で見ている人間とみられた人間が向き合う状態を作る.

\begin{figure}[tbp]
  \centering
  \includegraphics[scale=0.6]{fig/OEBsideSlideimg.png}
  \caption{クリックイベント発生時の全天球ディスプレイ上の表示}\label{OEBGUI1}
\end{figure}

この時,例えば3人が120度の間隔でOmniEyeBallの周りに座っている状況を考えると,
顔を向けられていない2人は,ほとんどPC側の参加者の顔を視認できない.
このような場合,この2人には画面上部に小さなウインドウを表示し,そこに
PC側参加者の顔部分を表示する.顔部分を表示すると,その顔は大抵正面方向を向いており
,自身が見られていると錯覚する恐れがある.それを防ぐため,PC参加者の顔ウインドウ
の隣には,見られている参加者の顔ウインドウを表示させる.

また,このままではPC側の使用者が,特定の個人を見ていない場合
(例えば,全員に対して語りかける場合)を区別できない.このため,
特定の個人を見ていない場合は,全員に対して画面上部に
PC参加者のみの顔ウインドウを表示させる.クリックした人物の方向を向く
仕様では,常に前回のクリックの結果が残り,いずれかの人物の顔方向に
PC参加者の顔が向くことになる.この方向に対して,あえて顔ウインドウ
を顔映像に被せることによって,大きく表示される方の顔映像の視認性を下げ,
自身が見られているという錯覚を防ぐことが狙いである.

\begin{figure}[tbp]
  \centering
  \includegraphics[scale=0.6]{fig/SPcase.png}
  \caption{特定の個人を見ていない場合の表示}\label{OEBGUI2}
\end{figure}

以上で提案したシステムを用いて,従来のビデオ通話アプリケーション
を用いた場合よりもPCの使用者の視線情報がOmniEyeBallの使用者に正確に伝わることが最終的な目標である.
\section{技術的課題}ハードウェア部分の実装に関しては,全天球ビデオカメラ2つ,PC2台,及び
OmniEyeBall1台を用意すればよい.

一方,ソフトウェア部分の実装に関しては,いくつか準備の必要がある
要素が存在する.

まず,ビデオストリーミングの方法について考える必要がある.
UDP等を用いて映像情報の送受信を行う必要がある.しかし,フレーム落ちの
問題など,現在使用されているビデオ会議アプリケーションのような堅牢で
高速な通信を実現するためには,高度な技術と時間を要する.
そこで,映像送受信については,すでに利用されているビデオ会議アプリケーション
を用いることにした.

しかし,ただ映像を送受信していては,PC側の操作に合わせて映像を回転させる
といった処理が出来ない.PC側では全天球ビデオカメラの映像を加工した後に
送信を行う必要がある.あるいは,PC側で表示するUIにおいて,OmniEyeBall側の
全天周パノラマ映像をそのまま表示させるのではなく,やはりクリックした人物の
顔を正面に移動させるなどの必要がある.
ここで,加工した映像をキャプチャしてビデオ会議アプリケーションに
認識させる方法と,一方で,送信されてきた映像をキャプチャーして,
以下で説明する画像加工ライブラリで扱えるようにする方法を考えなければならない.

この両方を解決するために,今回は仮想ビデオカメラを用いる.
仮想ビデオカメラとは,OBS-virtual-cam\cite{7}やUnityCapture\cite{8}
等のような,特定のアプリケーションの映像をリアルタイムでキャプチャし,
その映像をあたかも接続したビデオカメラの映像のように扱えるものである.
これを用いることで,ビデオ会議アプリケーションや画像加工ライブラリ
が,加工済みの映像や送信されてきた映像をビデオカメラの映像として認識し,
ストリーミングや加工が出来るようになる.

\begin{figure}[tp]
  \centering
  \includegraphics[scale=0.7]{fig/flow.png}
  \caption{映像の加工と送受信のフロー}
\end{figure}

次に,映像の加工方法について述べる.映像の加工はpythonなどで利用可能な
画像加工ライブラリを用いた.同時にPC画面に表示するUIもpythonのライブラリで作成した.



\chapter{OmniEyeBall対PC間通信の実装}
%\section{本章の概要}
%\section{本研究の内容と提案手法}・ビデオストリーミングの実装の手間を省くため,画像送受信にはZoomとOBS studioを用いた

・見ている人をクリックすると,OEB側でその方向を向くことで,視線情報を伝えた

・顔が見えなくなる問題は,すべての人間に小窓を表示することで解決した

・また,小窓で正面を向いており,見られていると錯覚するのを防ぐため,隣には現在見られている相手を表示した

・特定の個人に話しかけていない状態を区別するため,そのようなときは全員にPC使用者のみの小窓を表示した

・PC側では,常に正面を向いていれば見たい人を正面に見据えられるように,クリックした人間の顔を正面に持ってくる処理を施した

・(以下4-2-1から5では,システムが動く様子を表した図や画像を用意する)

%\section{本研究の実装}
%  \subsection{360度カメラ}・RICHO THETA V

・RICHO THETA S
%  \subsection{OmniEyeBall}・(OmniEyeBall)4K画質球体ディスプレイを使用

・glomal360

%  \subsection{オンラインビデオストリーミング}・送受信にはzoomを用いた

・加工後や,加工前の映像を仮想カメラ映像としてキャプチャするため,OBS virtual camを用いた

・(OEBにうつす映像)THETA Sで撮影した映像をOpenCVで加工->OBSでキャプチャ->zoom

・(PCに移す映像)THETA Vで撮影した映像をZoomで送信->OBSでキャプチャ->OpenCV + TKinterで加工
%  \subsection{OpenCVを用いた360度パノラマ画像の処理}・OBS studioでキャプチャ

・OpenCV4で加工

・OEB用の画像は,パノラマ画像を極座標変換して作成

・PC用の画像は,OpenCVで不要な領域を削除

%  \subsection{GUIアプリケーション}・TKinterを使用

・クリックイベントを追加,クリックした場所で誰を見ているか判断し,真ん中に持ってくる

・上のクリックイベントと同時に,OpenCVで小窓を作成
  \section{ハードウェア実装}%・RICHO THETA V

%・RICHO THETA S

%・(OmniEyeBall)4K画質球体ディスプレイを使用

%・glomal360

\subsection*{全天周ビデオカメラ}
全天周ビデオカメラとして,PC側ではRICOH THETA S,
OmniEyeBall側ではRICOH THETA Vを用いた.

\subsection*{球体ディスプレイ}
全天球動画の出力装置としては
渋谷光学の球体プロジェクターであるGlomal350を使用する.
  \section{ソフトウェア実装}\subsection*{映像ストリーミング}

\chapter{実験}
%\section{本章の概要}
\section{実験の目的}%・OEBと4章で紹介した視線情報共有手法を用いて,一人が多人数に対しプレゼンテーションを行う際,被検者がどのように感じたかを定性的に調べる

%・意図した視線が正確に伝わっていたか,定量的な評価も行う

%・定量的データとアンケート結果から,本システムの優れた点と改善すべき点を明らかにし,今後の研究へとつなげる

OmniEyeBall及び,4章で紹介した視線共有手法を用いて,1人が多人数に対しプレゼンテーションを行う際に
被検者がどのように感じたかを定性的に調べるのが本実験の目的である.

一方で,意図した視線が正確に伝わっていたかを,視線情報の履歴から算出し,定量的な評価も行う.

定量的データとアンケート結果から,本システムの優れた点と改善すべき点を明らかにし,今後の研究へとつなげる.
\section{実験用アプリケーション}・(4章と重複?要修正)

・画面のレイアウトについて詳しく記述する

\section{実験手順}
  \subsection{実験設定}
%・プレゼンター1人,傍聴者3人によるプレゼンテーションを行った

%・プレゼンと質問時間合わせて10分程度

%・提案アプリケーション+OEB使用時と,Zoomのみを使用した場合の2度行った

%・プレゼンター負担削減のため,プレゼン資料は事前に作成した(2種類)

%・また,各傍聴者に,最低一回の質問機会を設けるため,各プレゼンに対し3つの質問項目を事前に用意した

%・実験環境は以下の通り

%・共通:THETA VとTHETA Sを使用,プレゼンターは全天球カメラに対し30cmほど離れていた,プレゼンターのPC詳細(後で調べる)

%・Zoom使用時:質問者側はMac book pro(栗岡机上:後で詳しく)を使用(横に3人並んで一つの画面を見た)

%・OEB使用時:質問者側はOmniEyeBallを使用

プレゼンター1人,傍聴者3人によるプレゼンテーション実験を行った.

プレゼンテーション,及び質疑応答の時間合わせて10分程度の時間を設けた.

従来の平面ディスプレイでの表示を用いた実験,及び5-2節で説明したアプリケーション
を用いた実験の2通りを行った.本来はカウンターバランスをとるため,逆順で2回行う予定だったが,
コロナ感染対策のため参加人数を最小限に抑える必要があり,今回は1回での実験となった.

プレゼンターの負担の削減のため,プレゼンテーションのための資料は事前に作成した.
2通りの実験のため,2種類の資料が存在する.プレゼンテーションの内容は,
SNS用の宣材写真として東京工業大学の敷地内で何を写真に撮るのか,という題で
本館,または図書館を紹介するという物であった.

\begin{figure}[tp]
  \centering
  \includegraphics[scale=0.7]{fig/slide1.png}
  \includegraphics[scale=0.7]{fig/slide2.png}
  \caption{プレゼンターに用意したスライドの一部}
\end{figure}

また,各傍聴者には,それぞれ最低1回の質問機会を設けるため,各プレゼンテーション
に対して3つの質問項目を事前に用意した.

実験環境は以下の通りであった.

\begin{itemize}
  \item プレゼンターの全天球カメラとしてTHETA S,傍聴者側にはTHETA Vを用いた
  \item プレゼンターと傍聴者は全天球カメラから約40cmほど離れていた
  \item (プレゼンターと傍聴者側で用いたPCの詳細を後で書く)
  \item 傍聴者は平面ディスプレイを使用する際は,PCの前に設置した全天周カメラの前で横に並んで座っていた
  \item また,OmniEyeBallを使用する際は,120°の間隔で,球状ディスプレイを囲んで座っていた
\end{itemize}

(ここに実験の様子を映した画像を配置する)


  \subsection{実験項目}・定量的な項目と定性的な項目それぞれ設けた

・以下定量的項目

・プレゼンター側が誰を見ていたのかをリアルタイムで記録

・同時に,質問者側は,目が合っていると感じる間,キーを押しっぱなしにして時間を記録

・見ているときに目が合ったと感じた割合(真陽性率)

・見ていないのに目が合ったと感じた割合(偽陽性率)を算出

・以下定性的項目

・System Usability Scale に基づいた5段階評価項目10個

・独自の評価項目(5段階評価2つ+記述式項目)

・OEBを使用したケースとPCを使用したケース両方を経験した被検者に対し,どちらの方が好ましいかを問う
記述式項目を設けた
  \subsection{被験者の動き}・OEB使用時と平面ディスプレイ使用時の2回の実験を行った(プレゼンター1人,質問者3人)

・平面ディスプレイ使用の実験の後,10分程度の休憩時間を設け,OEBを使用する実験に移った

・実験の開始前に,使用するアプリケーションの操作説明を行った

・また,5分程度,事前準備したスライドの内容をある程度覚えてもらった(プレゼン中,出来るだけ映像の方を見てもらうため)

・プレゼンターは,プレゼン中,3つ存在するプレゼン内容のチャックポイントに合わせ,ランダムな順番で1人づつ,対応する被験者を見てもらった

・また,質問中は質問を受けている被検者の方向を見るように指示した

・OEB使用時は,見る被験者の映像をクリックし続けることで,視線情報のログを取った

・平面ディスプレイ使用中は,被検者それぞれに対応したキー(映像に移っている左の人間からA,S,D)を押しっぱなしにしてもらうことで視線情報のログを取った

・一方,質問者側の被験者3人には,プレゼン中はプレゼンを聴くように指示した

・質問は,事前に準備した質問3つをランダムな順番で行った(プレゼンターから質問するように促される)

・上記の質問が終わったのちは各質問者から自由に質問してもらい,プレゼンと質問合わせて約10分使用した
\section{実験結果}
  \subsection{定量的評価}定量的評価の結果を以下に示す.

\begin{table}[tp]
  \begin{center}
  \begin{tabular}{|c|c|c|}
  \hline
       & 真陽性率 & 偽陽性率 \\ \hline
  被検者1 & 0.05 & 0.02 \\ \hline
  被検者2 & 0.57 & 0.12 \\ \hline
  被検者3 & 0.00 & 0.00 \\ \hline
  \end{tabular}
  \caption{平面ディスプレイ使用時}\label{rate1}
\end{center}
\end{table}

\begin{table}[tp]
  \begin{center}
  \begin{tabular}{|c|c|c|}
  \hline
       & 真陽性率 & 偽陽性率 \\ \hline
  被検者1 & 0.06 & 0.07 \\ \hline
  被検者2 & 0.87 & 0.28 \\ \hline
  被検者3 & 0.19 & 0.00 \\ \hline
  \end{tabular}
  \caption{OEB使用時}\label{rate2}
\end{center}
\end{table}

  \subsection{定性的評価}定性的評価の,5段階評価項目の結果は以下の通り.
記載した箱ひげ図については,左から質問1から12の順に並んでいる.
\begin{figure}[tp]
  \centering
  \includegraphics[scale=0.8]{fig/boxplot11.png}\label{boxplot1}
  \caption{平面ディスプレイ使用時の5段階評価項目結果}
  \includegraphics[scale=0.8]{fig/boxplot10.png}\label{boxplot2}
  \caption{OmniEyeBall使用時の5段階評価項目結果}
\end{figure}



\begin{table}[tp]
  \begin{center}
  \begin{tabular}{|c|c|c|}
  \hline
       & 平面時  & OEB時 \\ \hline
  質問1  & 2.75 & 3.5  \\ \hline
  質問2  & 2.5  & 3.25 \\ \hline
  質問3  & 3.5  & 3.25 \\ \hline
  質問4  & 1.75 & 3.75 \\ \hline
  質問5  & 3    & 3.5  \\ \hline
  質問6  & 3.5  & 3.25 \\ \hline
  質問7  & 4.75 & 3.75 \\ \hline
  質問8  & 2.25 & 1.5  \\ \hline
  質問9  & 3.5  & 3.5  \\ \hline
  質問10 & 1.5  & 2.5  \\ \hline
  質問11 & 1.75 & 3.25 \\ \hline
  質問12 & 4    & 3.25 \\ \hline
  \end{tabular}\label{ave}
  \caption{各質問の4人の平均値}
\end{center}
  \end{table}

記述式項目については5-5節にて言及する.

また,平面ディスプレイを使用したケースとOmniEyeBallを使用したケースの
どちらが好ましいかを問う質問では,前者が好ましいと答えた人数は1人,
後者が好ましいと答えた人数は3人であった.
\section{実験結果から得られる知見と考察}\begin{comment}
・表5.1,5.2を見るとOEB使用時,平面ディスプレイ使用時共に真陽性率(プレゼンターが見ていた時間の中で,目が合ったと感じた時間の割合)
の最大値,最小値の差が非常に大きい(極端な例は平面ディスプレイ使用時の0.00(一度もキーを押さなかった))

・自由記述意見の1つ「発表者の顔の画像が暗くて小さいため、何となく正面を向いている時に目があっていそうだと感じた。曖昧であり、判断に困った」(平面ディスプレイ使用時)
に見られるように,そもそも「目が合っている」という感覚自体があいまいで,人によって見られていると感じた機会が不ぞろいである原因であると考えられる.

・真陽性率は3人とも平面ディスプレイ使用時に対し,OEB使用後の方が増加している

・サンプル数の少なさ,及び順序効果により,比較はできないが,本実験で使用したアプリケーションによって
視線情報が共有されやすくなったという仮定に対し,一考の余地がある.

・一方で,視線情報を小窓の表示状態などからも判断できるOEB使用時の方が,偽陽性率が下がると予想していたが,
2者はOEB使用時の方が偽陽性率が増加した.特に,被検者2の偽陽性率は0.16上昇し0.28となっている.

・「正面に発表者がいて、さらに窓に発表者がいることに違和感を感じました。」(OEB使用時)という意見があった.
本アプリケーションは,「特定の個人を見ていない場合,全員に小窓を表示する」という仕様になっている.
この仕様では,前回の人物クリックの結果,質問者側でプレゼンターの顔が正面に移動したのにもかかわらず,その上に小窓を表示され
2つの顔が映ってしまう.そのため,見られていない時も見られていると錯覚した可能性がある.

・OEBを使用する場合において,上記以外にも小窓の表示方法に関する意見が多くみられた.

・「顔の切り抜きはもう少し下に置くことは可能か?」

・「OEBについての説明が少ないので仕様がよく分からなかった。小窓がなくなった時がよくわからなかった。小窓が唐突に消えるのに驚いた。」

・よって,小窓の表示方法に関しては再考する必要がある.

・また,全天球ビデオカメラにおいて,被写体の顔が小さく表示されるという意見も多くみられた.

・「3人のうち誰か一人を見るということが難しいと感じた(視野に全員が収まっているので誰か一人にフォーカスして話しづらい)」

・「顔とカメラが遠い?からなのか、そもそも目が見えなかった」

・「発表者の顔の画像が暗くて小さいため、何となく正面を向いている時に目があっていそうだと感じた。曖昧であり、判断に困った。」(いずれも平面ディスプレイ使用時)

・全天周カメラ自体の視野の広さと,撮影領域上下が広がるような歪みによって,相対的に顔が小さくなることが原因であると考えられる.

・よって,全天周カメラを用いて顔を映したコミュニケーションを行う際には,何らかの方法で顔部分をクローズアップする必要がある.

・「ディスプレイのサイズ的に3人とも常に視野の中に収まってはいるので、クリックした人が真ん中に来る必要はないかな…と感じた。
画像そのものが移動するよりは、クリックしている間はカーソルの周辺に枠がでる…とかのほうが「ひとりにフォーカスしている感」があって良いかもしれないと思いました。」(OEB使用時,プレゼンター側)

・PCを使用するプレゼンター側において,常にカメラの方を向くことで見たい人間を正面に捉えるように設計したが,常に3人とも視界に入っているためかえって混乱を招くことになった.
\end{comment}

\subsection*{定量的データの考察}

\subsubsection*{目が合うという感覚の曖昧さ}

\ref{rate1}と\ref{rate2}を見ると,OmniEyeBall使用時,平面ディスプレイ使用時共に
真陽性率の最大値と最小値の差が非常に大きいことが分かる.極端な例としては
平面ディスプレイ使用時に$0.00$という値が見られ,これは実験中に1度もキーを
押さなかったことを表している.

自由意見の一つに
\begin{itemize}
  \item 発表者の顔の画像が暗くて小さいため、何となく正面を向いている時に目があっていそうだと感じた。曖昧であり、判断に困った.(平面ディスプレイ使用時)
\end{itemize}
というものが見られた.今回,各個人の「目が合っている」という感覚に重点を置き,あえてそのままの形で
計測を行ったが,この結果にあらためて「目が合っている」と感じる状況そのものが
個人によって如何に異なるかが表れている.

\subsubsection*{真陽性率と偽陽性率について}

真陽性率については,3人とも平面ディスプレイ使用時に対し,
OmniEyeBall使用時の方が増加している.サンプル数の少なさ,
及び順序効果により,比較は出来ない.しかし,本実験で使用したアプリケーション
によって,視線情報が共有されやすくなったということは一考の余地がある.
\begin{itemize}
  \item 明示的に「話しかける対象を指定する」という動作をする必要があるので、モニターに複数人が映されているという話しかける相手を意識しづらい状況で、より今誰に話かけているかを強く意識できたのが良かった
\end{itemize}
というようにプレゼンターからのコメントも見られ,視線それ自体を意識する機会を向上させた可能性もある.

一方で,OmniEyeBall使用時の偽陽性率も,平面ディスプレイ使用時の
それ以上となっている.特に,被検者2の偽陽性率は0.16上昇し0.28となっている.
偽陽性率は,見られていないのに見られていると
錯覚した時間の割合であり,視線情報が正しく伝わっていないことを示す.
自身が見られている場合と見られていない場合で表示方法が異なる今回の仕様では
予期していない結果であった.この原因としては「特定の個人を見ていない場合,全員に小窓を表示する」
という仕様にあると考える.この仕様では,前回の人物クリックの結果,質問者側でプレゼンターの顔が正面に移動したのにもかかわらず,
その上に小窓を表示され2つの顔が映ってしまう.
全天球カメラの映像のままでは,見られていると錯覚するため,あえて小窓を上に被せた.
しかし,自由記述欄には以下のコメントが存在した.
\begin{itemize}
  \item 正面に発表者がいて、さらに窓に発表者がいることに違和感を感じました。(OmniEyeBall使用時)
\end{itemize}
加えて
\begin{itemize}
  \item OEBについての説明が少ないので仕様がよく分からなかった。小窓がなくなった時がよくわからなかった。小窓が唐突に消えるのに驚いた。(OmniEyeBall使用時)
\end{itemize}
このように,小窓の表示方法が複数存在し,かつ直感的でなかったことが原因であったと考えられる.

\subsection*{定性的データの考察}
5段階評価項目は,奇数番目の項目が肯定的な,偶数番目の項目が
否定的な項目となっている.サンプル数は4であり,比較は出来ない.
(実際に各質問事項に対し,対になっているデータのt検定を
有意水準$5\%$で行ったところ,p値の最小は
質問11の0.058であり,有意差は見られなかった.)

だが,質問事項4,11のように,回答に偏りがあるものも見られ,
これらから本アプリケーションの優れていた点と改良すべき点を模索する.

\subsubsection*{対面感についての考察}
質問11の結果に注目すると,平面ディスプレイの時は1又は2
であった回答がOmniEyeBall使用時には3又は4にまで上昇している.
これは,実験用アプリケーションを用いることで,ビデオ会議の参加者と
あたかも対面しているかのような感覚を増させることが出来たことを
示唆している.実際に,以下のようなコメントが見られた.
\begin{itemize}
  \item 発表者の見え方が対面の時に近くて、直感的であったため。また、質問者と発表者が会話している様子がわかりやすかったから。(記述式項目2の回答)
\end{itemize}

\subsubsection*{会話状況の把握についての考察}
5段階評価項目としては設けていないが,記述式項目2の回答に
会話状況が把握しやすくなったという旨の記述が4人中2人に見られた.
\begin{itemize}
  \item 発表者の見え方が対面の時に近くて、直感的であったため。また、質問者と発表者が会話している様子がわかりやすかったから。
  \item 今誰に対して話しているのかが明確になってわかりやすいから. 話聞いていなくて誰に対して話しているか分からない時に,即座に誰と話しているかどうかが認識できるので良いと思った.
\end{itemize}

\subsubsection*{アプリケーションの仕様の複雑さの考察}
質問4の結果を見ると,平均値に大きな差が出ており,OmniEyeBall
を使用するケースの方が否定的な結果になっている.質問4はアプリケーション
の仕様の分かりやすさについての項目である.自由記述のコメントには
本アプリケーションの仕様の複雑さの指摘が2つ見られた.
\begin{itemize}
  \item 正面に発表者がいて、さらに窓に発表者がいることに違和感を感じました。
  \item OEBについての説明が少ないので仕様がよく分からなかった。小窓がなくなった時がよくわからなかった。小窓が唐突に消えるのに驚いた。
\end{itemize}
どちらも小窓の表示方法に関する指摘である.
小窓の表示の改良は喫緊の課題であるといえる.

\subsubsection*{全天球カメラの視野に関する考察}
最後に,記述式項目の回答に,全天球カメラの映像
それ自体に関わるものがいくつか見られた.
\begin{itemize}
  \item 発表者の顔の画像が暗くて小さいため、何となく正面を向いている時に目があっていそうだと感じた。曖昧であり、判断に困った。(平面ディスプレイ使用時)
  \item 3人のうち誰か一人を見るということが難しいと感じた(視野に全員が収まっているので誰か一人にフォーカスして話しづらい(プレゼンター)
\end{itemize}
以上のように,ビデオ会議の参加者が小さく表示されてしまうという問題がある.
全天球ビデオカメラの視野の広さと相対的に,被写体が小さくなってしまったり,
あるいは正距円筒図法の特徴として,カメラ上・下部の映像ほど拡大されてしまうせいで,
中心付近に移る顔画像が相対的に小さくなってしまうことが原因として考えられる.

\subsection*{本節のまとめ}
以上の考察から,本システムの優れている可能性のある点として以下が挙げられる.
\begin{itemize}
  \item 視線の情報を伝えたり,得ようとする機会が向上する
  \item 立体的な映像から,対面しているかのような感覚を得やすい
  \item 会話状況の把握がしやすくなる
\end{itemize}

一方,改善すべき点としては以下が挙げられる.
\begin{itemize}
  \item 小窓の表示方法の改善
  \item PC側で表示する全天球パノラマ映像の顔部分の表示
\end{itemize}















\chapter{考察}
%\section{本章の概要}
\section{本システムの問題点}\begin{comment}
・また,SUSのネガティブな評価項目がOEB使用時で有意差は見られないものも,5に近くなってしまっている.

・OEB使用時のアンケート結果でGUIに関する意見が多く出ている.

・特に「仕様が分かりづらい」,「小窓の表示位置を変えたい」という趣旨の意見が多くみられた.

・小窓の位置のみならず,本来の全天球カメラの映像に被さって小窓が表示される仕様など,映像が複雑になってしまう状況もあり,
それが混乱を招いた可能性がある.

・また,全天球カメラ自体の視野の広さから,パノラマ映像では顔が小さく表示されやすく,PCの画面に表示する
全天周カメラの映像表示方法にも改善の余地がある.

・算出した偽陽性率に改善が見られなかった点から,小窓に表示する顔の表示方法自体も変更する余地がある.

・実験結果に関係しない事項としては,実験用アプリでは被験者の位置が変更しない制限があったが,
実際は被写体は自由に動くことが出来るべきである.被写体が移動しても正しく小窓を表示できるように
する必要がある.
\end{comment}
5章で,立体的な映像から,対面しているかのような感覚を得やすいという結論
が得られた.これに基づいて,実際に対面した時に見えるような立体視を
再現することが効果的であると考えられる.6章ではより立体感のある
小窓の表示方法や,全天球カメラの映像の表示方法を組み込んだアプリケーションを
提案する.

加えて,小窓の表示位置の改善を行う.そうすることで,よりユーザーに対して
分かりやすい映像の表示方法を提案する.

実験結果に関与しないが,改善すべき点として,実験と違い
被写体が移動するという条件が加わる.被写体が移動しても
動的に小窓の表示位置を変更できるように,顔認識ライブラリを
用いた手法を組み込む.

\section{本システムの応用}
  \subsection{対話相手表示形式}%・6-2-1~3節では小窓の表示方法,あるいは顔自体の表示方法などを変更したアプリケーションを提案する

%・まず,実験用アプリケーションと同様に,

\begin{comment}
\begin{itemize}
  \item 6-2-1~3節では小窓の表示方法,あるいは顔自体の表示方法などを変更したアプリケーションを提案する
  \item まず,実験用アプリケーションと同様に,小窓に,PC使用者と,見られている参加者の2人を表示する手法を提案する
  \item 実験用アプリケーションと異なる点は以下の通り
  \item 小窓の表示位置を,球体ディスプレイの赤道部分に寄せる
  \item 顔認識ライブラリdlibを用いて,顔を認識した位置に対して小窓を表示する
  \item PC使用者は,実験用アプリケーションの時と異なり,指定の位置ではなく可変な被検者の顔の位置をクリックすることで
        被験者を見ることが出来る
  \item 顔を中心に持ってくるのではなく,GUIの空いていた領域(画面下部)にクリックした人物の顔の切り取りを表示する
  \item (dlibについての説明,アプリケーションの様子,クリックした位置に顔を持ってくるプログラムについての説明を行う)
\end{itemize}
\end{comment}

この節では,小窓にPC使用者と見られている人間の2つの顔を
表示するアプリケーションを提案する.実験用アプリケーション
との変更点は以下のとおりである.
\begin{itemize}
  \item 小窓の表示位置を,球体ディスプレイの赤道部分に寄せる
  \item 顔認識ライブラリDlib\cite{12}を用いて,顔を認識した位置に対して小窓を表示する
  \item PC使用者は,実験用アプリケーションの時と異なり,指定の位置ではなく可変な
  被験者の顔の位置をクリックすることで被検者を見ることが出来る
  \item 顔を中心に持ってくるのではなく,GUIの空いていた領域(画面下部)にクリックした人物の顔部分を表示する
\end{itemize}

変更点の1つ目は
\begin{itemize}
  \item 顔の切り抜きはもう少し下に置くことは可能か?
\end{itemize}
という,被検者からのコメントに対しての実装である.実際に,
正距円筒図法によって出力された全天球パノラマ画像に,正方形で切り取った
顔部分の画像を張り付けると,歪みを考慮していないために,上部分ほど
実際より小さく表示されてしまうという問題がある.
例えば緯度60°に相当する部分は,正距円筒図法上では赤道と同じ長さであるが
球体ディスプレイ上では赤道の長さの半分になってしまう.(表\ref{projection1}参照)よって,歪みの
影響の少ない球体ディスプレイの赤道近くに顔の小窓を表示することが適切であると
考えた.
\begin{figure}[tp]
  \centering
  \includegraphics[scale=0.7]{fig/projection1.png}\label{projection1}
  \caption{上部分が縮小して表示される様子}
\end{figure}

変更点の4つ目は
\begin{itemize}
  \item ディスプレイのサイズ的に3人とも常に視野の中に収まってはいるので、クリックした人が真ん中に来る必要はないかな…と感じた。画像そのものが移動するよりは、クリックしている間はカーソルの周辺に枠がでる…とかのほうが「ひとりにフォーカスしている感」があって良いかもしれないと思いました。
  \item 3人のうち誰か一人を見るということが難しいと感じた(視野に全員が収まっているので誰か一人にフォーカスして話しづらい)
\end{itemize}
というプレゼンターからのコメントを反映させた.
また,顔にフォーカスし,表示させることで,全天球カメラの視野の広さによって
顔が相対的に小さく見えてしまう問題の解決も目指した.

\subsection*{Dlib}

\begin{figure}[tp]
  \centering
  \includegraphics[scale=0.7]{fig/dlib.png}
  \caption{Dlibの顔検出} \cite{13}
\end{figure}

顔認識には,C++やpythonで利用可能な,クロスプラットフォームソフトウェアライブラリ
であるDlibを用いた.Dlibには様々な種類の顔認識モデルが事前に準備されている.
代表的なものを以下に示す.
\begin{itemize}
  \item HOG(Histograms of Oriented Gradients)+SVN(Support Vector Machine)
  \item CNN
\end{itemize}

\subsubsection*{HOG+SVN}
HOGは局所領域の輝度の勾配を
求め,ヒストグラム化した特徴量である.画像はブロックと呼ばれる局所領域
に分割され,さらにブロックはセルと呼ばれる局所領域に分割される.セル毎に
各ピクセルの輝度勾配を計算し,勾配の方向で分類したヒストグラムを作成する.
各セルの各勾配方向におけるヒストグラムは,ブロック内のヒストグラムの総和に
よって正規化される.

SVNは,データを2クラスに分類するために
最適な超平面を求めるアルゴリズムである.$n$次元空間における超平面は,
次元が$n-1$の平坦な空間であり,元の空間を2つに分割する.

\subsubsection*{CNN}
CNNは,従来のニューラルネットワークに,畳み込み層やプーリング層といった
層を組み合わせて作られるニューラルネットワークである.画像認識の分野では
よく用いられている.

\begin{figure}[tp]
  \centering
  \includegraphics[scale=0.7]{fig/CNNep.png}
  \caption{CNNの一例} \cite{13}
\end{figure}

畳み込み層では,学習可能なパラメータであるカーネル及びバイアスを用いて,入力に対して
畳み込みが行われる.画像データの畳み込みに際しては,ある注目ピクセルと
その周辺のピクセルが出力に用いられることで,画像の空間的な情報が損なわれにくい
という利点がある.

カーネル$K$のサイズが$N*N$であった場合,画像のあるピクセルの周辺$K*K$サイズの
領域$A$にカーネルを適用した結果は,$K$と$A$のアダマール積となる.すなわち

\begin{eqnarray}
  \sum_{i=1}^{N} \sum_{j=1}^{N} k_{ij}a_{ij} \nonumber 
\end{eqnarray}
である.

カーネルを適用する領域は通常1ピクセルずつずらされる.そうして
全ての領域にカーネルが適用され,バイアスを加算した結果が畳み込み層の出力
として用いられる.

\begin{figure}[tp]
  \centering
  \includegraphics[scale=0.6]{fig/conv.png}
  \caption{畳み込みのイメージ} \cite{14}
\end{figure}

プーリング層は,通常畳み込み層の直後の層であり,畳み込み層の
出力を,カーネルを用いてダウンサンプリングした結果を出力する.

ここで用いられるカーネルは,カーネル適用範囲の平均値を出力するものと,
カーネル適用範囲の最大値を出力するものがよく用いられている.
前者の場合はAverage Pooling層と呼ばれ,後者ならMax Pooling層と呼ばれる.
カーネルの適用範囲のずらし幅はストライドと呼ばれ,様々な値が設定されている.

\begin{figure}[tp]
  \centering
  \includegraphics[scale=0.7]{fig/pooling.png}
  \caption{プーリングのイメージ} \cite{14}
\end{figure}

今回のアプリケーションでは,より精度の高いCNNを用いたモデルを使用した.
一方で,HOG+SVNを用いるモデルは,識別実行速度が速いという利点がある.






  \subsection{横顔生成形式}\begin{comment}
\begin{itemize}
  \item 2人の顔をウインドウで表示する代わりに,対面で人の横顔を見るような表示
  \item 横顔を表示することで,立体感を拡張させ,あたかも対面しているかのような状況を作り出す
  \item 今回は横顔の生成は実際の映像を用いては行っていない
  \item Unity上で3Dモデルを横から撮影し,UnityCapture\cite{8}を用いて仮想カメラの映像としてOpenCVで処理した
  \item 後述するおかしら会議\cite{10}のように,Unity内の球体に画像をマッピングして,疑似的に顔を生成し,横から撮影する
  というった方法も考えられる.
  \item 機械学習ベースの横顔生成手法としてYiboら\cite{11}の方法が挙げられる.
\end{itemize}

\begin{figure}[tp]
  \centering
  \includegraphics[scale=0.6]{fig/PCimgSlideYoko.png}
  \caption{OmniEyeBall上で,PC使用者の表示位置が移動する模式図}
\end{figure}
\end{comment}

6-2-1で会話中のビデオ会議参加者を表示する方法を提案した.本節では
立体的な映像を表示するという部分に着目し,小窓に横顔を
表示する方法を提案する.

\begin{figure}[tp]
  \centering
  \includegraphics[scale=0.7]{fig/yokogao.png}
  \caption{横顔が見える状況} \label{yokogao}
\end{figure}

表\ref{yokogao}のように,複数人で対面している状況を考える.
話者が自身に話しかけていない場合,別の人物に顔を向けている
場合が多く,その際には話者の横顔が視認できる.
ここでは,小窓に話者の正面顔ではなく横顔を表示する
方法を提案する.対面時の立体的情報を再現することで,
ビデオ会議の参加者に更なる没入感を提供することが目的である.

実装時間の関係で,参加者の実際の顔の横顔を再現するには至らなかった.
このことは今後の課題とする.代用案として,昨今様々な場所で配布されており,
入手・仕様・作成が容易な3Dモデルを利用して,横顔を表示する手法を提案する.

3Dモデルの撮影にはUnity\cite{16}を用いた.UnityではOpenCV及びDlib
が利用可能で,Dlibで取得した顔のランドマーク情報から3Dモデルの体や表情を
動かすことが可能である.また,Unityでは自由にシーン描画カメラを配置することができる,
ここでは3Dモデルの左右にカメラを配置し,リアルタイムで横顔を撮影した.
Unity内カメラの映像はUnityCapture\cite{8}を利用することで,容易に
仮想カメラの映像として利用することが出来る.

\begin{figure}[tp]
  \centering
  \includegraphics[scale=0.7]{fig/unitycam.png}
  \caption{3Dモデルを撮影する様子}
\end{figure}

\begin{figure}[tp]
  \centering
  \includegraphics[scale=0.6]{fig/unitycapture.png}
  \caption{UnityCaptureを用いた他アプリケーションへの映像ストリーミング}
\end{figure}

通信の高速化のため,プロセス間通信で顔の位置情報を送信し,カメラを移動させる
ということは行わなかった.OmniEyeBallの使用者から見て,右側の参加者が見られているときには
左から見た横顔が,左側の参加者が見られているときには右から見た横顔が表示されるようにした.

\begin{figure}[tp]
  \centering
  \includegraphics[scale=0.4]{fig/yokoV.png}
  \caption{生成した横顔をOmniEyeBall上で表示させた様子}
\end{figure}

今後,ビデオ会議参加者自身の顔を生成する方法としては,
後述するおかしら会議\cite{10}の表示法を用いる手段がある.
Unity内の仮想球体にspriteとして,PC使用者の顔写真を張り付ける.
そうした球体を撮影することで,疑似的な顔写真を生成する.

機械学習ベースの方法として,横顔生成手法としてYiboら\cite{11}の方法が挙げられる.
これらの方法を用いた表示は,今後の研究課題とする.
  \subsection{おかしら会議形式}\begin{itemize}
  \item 全天球映像の表示方法自体を変更する手段として,宮藤ら\cite{10}のおかしら会議が挙げられる
  \item 球体ディスプレイそのものを顔として見立て,全天球映像の顔部分のみをマッピングする
  \item マッピングする際に,全天球パノラマ画像上下部分に見られる歪みを考慮し,修正する式を用いている
  \item (式を乗せて,どのように歪み補正が行われているかを説明)
  \item (アプリケーションが動いている様子も載せる)
\end{itemize}
\section{今後の展望}\begin{comment}
\begin{itemize}
  \item 6-2で,様々な顔の表示方法を提案した.
  \item コロナ禍の影響で十分な実験を行うことが出来なかった
  \item 提案したアプリケーションを用いて,十分な実験を行い,比較を行うことで,
  最適な表示方法を決定することが今後の課題である
  \item 一方で解決できていない問題も存在する
  \item 小窓が表示される際に,突然表示されて使用者が驚いてしまう
  \item 複数の人間に話しかけている場合の顔や小窓の表示方法
  \item 今後の研究でそれらの問題を解決する方法を模索していく
  \item また,今回のアプリケーションは個人対複数人といったユースケースに限られている
  \item 個人の参加者が複数人いた場合,OmniEyeBall上での表示をどのようにするかという問題がある.
  \item 例えば,OmniEyeBall側で,現在映したい人物を選択できるようにすれば,今回のシステムをそのまま活用できる
  \item 一方で,会話にPC使用者が複数人参加している場合には,同時に映すことが出来ない
  \item このようなユースケースにも対応していく方法も検討する必要がある
\end{itemize}
\end{comment}
本章では,5章で得られた知見に基づいていくつかのアプリケーションを
提案してきたが,依然解決できていない問題がある.

\subsubsection*{特定の一人を見ていないケース}
アンケート結果によって,小窓を既に表示されている顔の上に表示するのは
混乱を招いている可能性があると結論付けた.よって提案したアプリケーションでは
その表示方法を廃止した.しかし,先の表示方法は,実験用アプリケーションで特定の一人を見ていないケース
を区別していた表示方法であった.この状況を表す代替の表示方法を提示する必要がある.

案の一つとしては,テキストや印を表示する方法がある.
しかし,小さいとサインが見えにくく,大きすぎると
映像の邪魔になってしまう.なおかつ,本アプリケーションは
従来の方法に比べ,SUSの評価が向上したとは言えず,これ以上の
情報提示は,ただでさえ複雑なアプリケーションをさらに複雑にすると
考え,実装を見送った.

\subsubsection*{全天球ビデオカメラの回転の問題}

本アプリケーションは,使用開始前の
状態として,PCの使用者が全天球パノラマ画像の
中心部を見ている前提で,画像のスライドや顔認識した位置に
画像を張り付ける処理を行ってきた.しかし,カメラが回転すると,
基準としていた位置がずれ,ずらす処理にも変更を加える必要がある.
だが,本研究では回転を考慮した処理を行っていない.

これを解決する案として,特徴点マッチングを用いる方法が考えられる.
特徴点マッチングによって,画像のスライド距離を算出し,その値を
本論文の処理に加えて反映させれば,この問題は解決できるであろう.

\begin{figure}[tp]
  \centering
  \includegraphics[scale=0.7]{fig/matching.png}
  \caption{特徴点マッチング} \cite{17}
\end{figure}

\subsubsection*{様々なケースへの対応}
今回は個人対複数人でのビデオ会議という状況
を前提において研究を進めてきた.しかし実際には
他にも以下のようなケースが存在する.
\begin{itemize}
  \item 複数のOmniEyeBallを使用するケース
  \item OmniEyeBallから孤立した参加者が複数人であるケース
  \item 孤立した参加者の何人かが共通の空間にいるケース
\end{itemize}

例えば2番目のケースは,OmniEyeBall側の参加者が現時点で注目したい
遠隔参加者を1人選ぶというようにする.さすれば,その1人に対して
今回提案したシステムが利用可能である.

しかし1番目のケース,3番目のケースは,複数人が同じ空間を
共有している.今回のような画像の回転を用いると,意図していない人物の
回転を同時に招いてしまう.これらの複雑なケースに沿ったシステムを
提案していくことは,今後の研究課題である.

\subsubsection*{まとめ}
今後の研究方針としては,先に述べたような
複雑なケースに対応するシステムの模索が考えられる.
一方,今回はコロナ禍の影響で,十分に実験が行えたとは言えない.
十分な時間を要して,今回提案した複数の表示方法について
綿密な実験を繰り返し,より多くのユーザーの実体験に
基づいた知見を獲得していくことも必要であろう.


%%%%%%%%%%%%%%%%%%%%%%%%%%%%%%%%%%%%%%%%%%%%%%%%%%

\chapter{結論}
結論は、網羅的にかつ簡潔に。

%%%%%%%%%%%%%%%%%%%%%%%%%%%%%%%%%%%%%%%%%%%%%%%%%%


%\appendix
%\chapter{定理1の証明}
%必要に応じて、付録を載せる。

%%%%%%%%%%%%%%%%%%%%%%%%%%%%%%%%%%%%%%%%%%%%%%%%%%

\backmatter
\chapter{謝辞}
本論文の執筆にあたり、議論して頂いた関係者に感謝する.

%%%%%%%%%%%%%%%%%%%%%%%%%%%%%%%%%%%%%%%%%%%%%%%%%%

\bibliographystyle{jplain}
\bibliography{references}
\begin{thebibliography}{99}
%  \bibitem{tokodai-xyz2015} 東工大太郎. 良い論文の書き方. \textit{Journal of XYZ}, Vol.~3, No.~4, pp. 15--34, 2015.
  
\bibitem{1}zoomの説明
\bibitem{2}google meetの説明
\bibitem{3}The GAZE groupware system: mediating joint attention in multiparty communication and collaborationの文献
\bibitem{4}Theta Vの説明(https://theta360.com/ja/about/theta/)
\bibitem{5}Collaboration in 360° Videochat: Challenges and Opportunitiesの文献
\bibitem{6}Can You See Me Now?: How Field of View Affects Collaboration in Robotic Telepresenceの文献
\bibitem{7}OBS-virtual-camの説明
\bibitem{8}UnityCaptureの説明(https://github.com/schellingb/UnityCapture)
\bibitem{9}OEBStudioの説明
\bibitem{10}おかしら会議の文献
\bibitem{11}Pose-Guided Photorealistic Face Rotationの文献
\bibitem{12}Dlibの説明(http://dlib.net/)
\bibitem{13}https://ci.nii.ac.jp/naid/40021907210
\bibitem{14}%https://www.jstage.jst.go.jp/article/hozen/advpub/0/advpub_1822/_pdf/-char/ja
\bibitem{15}glomal 350の説明(https://www.aisan.co.jp/products/glomal350.html)
\bibitem{16}Unity https://unity.com/ja
\end{thebibliography}

\end{document}
